\documentclass[11pt]{article}
\usepackage{amsmath}
\usepackage{caption}
\usepackage[margin=1in]{geometry}
\usepackage{graphicx}
\usepackage{subcaption}
\usepackage{mwe}
\usepackage{float}
\usepackage{multicol}

\title{Step 7: Experiments}
\author{Fatema Alkhanaizi}
\date{\today}

\begin{document}
    \maketitle
    All parameters were tested on the same node in Hamilton queue par6; the number of processors for the node was 32. The values reflect a single time step value for computing molecular dynamics for the given number of particles. Different numbers of molecules were tested, results in table-\ref{n100} to table-\ref{n10000}. The elapsed time was meatured using openMp function omp\_get\_wtime for both serial and parallel functions; to give a unified time reference; this could lead to minor errors when estimating the elapsed time. 
    \begin{figure}[H] 
        \centering
        \begin{minipage}{0.49\textwidth}
            \centering
            \includegraphics[width=1\textwidth]{{step7-sp}.png}
        \end{minipage}
        \caption{Speed up as number of cores increase}
        \label{sp}
    \end{figure}
    The plots follow Amdahl's law to some extent (line for N=10000) but breaks when the number of cores increase as parallelization causes a big overhead. The results for N=100, table-\ref{n100}, can be explained by taking into account the overhead introduced by the parallelization process. So, it is expected for N$<=$100, increasing the number of cores will result in a decrease in Speed. The serial code is the most fit for this setup. For N=1000 case, table-\ref{n1000}, the parallel code performed better as the number of cores doubled. To some extent, the parallel code for N=1000 case performed better than N=10000 case. However, the speed rate increased for the case where N=10000 case as the number of cores increased beyond 8 cores; the speed up slowly reaching a limit. 

\begin{table}[H]
    \centering
\begin{tabular}{|c|c|c|c|c|}
    \hline
    Version & Number of Processors & Time Elapsed (seconds) & Speed Up & Efficiency \\
    \hline
    Serial & 1 & 0.000110134 & 1.0 & 100\% \\
    \hline
    Parallel & 1 & 0.000120038 & 1.0 & 100\% \\
             & 2 & 0.000150815 & 0.80 & 40\% \\
             & 4 & 0.000194034 & 0.62 & 15.5\% \\
             & 8 & 0.000236925 & 0.51 & 6.4\% \\
             & 16 & 0.000313766 & 0.38 & 2.4\% \\
             & 32 & 0.0129332 & 0.0093 & 0.03\% \\
    \hline
\end{tabular}
\caption{Results for running a single time step for 100 particles}
\label{n100}
\end{table}

    \begin{table}[H]
        \centering
    \begin{tabular}{|c|c|c|c|c|}
        \hline
        Version & Number of Processors & Time Elapsed (seconds) & Speed Up & Efficiency \\
        \hline
        Serial & 1 & 0.0199244 & 1.0 & 100\% \\
        \hline
        Parallel & 1 & 0.0173634 & 1.0 & 100\% \\
                 & 2 & 0.0102346 & 1.70 & 85\% \\
                 & 4 & 0.00509077 & 3.41 & 85.25\% \\
                 & 8 & 0.00318944 & 5.44 & 68\% \\
                 & 16 & 0.0020295 & 8.55 & 53.43\% \\
                 & 32 & 0.0129332 & 1.34 & 4.19\% \\
        \hline
    \end{tabular}
    \caption{Results for running a single time step for 1000 particles}
    \label{n1000}
    \end{table}

    \begin{table}[H]
        \centering
        \begin{tabular}{|c|c|c|c|c|}
        \hline
        Version & Number of Processors & Time Elapsed (seconds) & Speed Up & Efficiency \\
        \hline
        Serial & 1 & 0.895323 & 1.0 & 100\% \\
        \hline
        Parallel & 1 & 0.836319 & 1.0 & 100\% \\
                 & 2 & 0.598624 & 1.38 & 69\% \\
                 & 4 & 0.318269 & 2.63 & 65.75\% \\
                 & 8 & 0.180262 & 4.64 & 58\% \\
                 & 16 & 0.102834 & 8.13 & 50.81\% \\
                 & 32 & 0.0835897 & 10.0 & 31.26\% \\
        \hline
        \end{tabular}
        \caption{Results for running a single time step for 10000 particles}
    \label{n10000}
    \end{table}

\end{document}